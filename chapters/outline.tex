\chapter{Outline}
\label{outline}

\citet{campbell16}

In this thesis we will examine the use of information and probability theory measures to track the pose, shape, and appearance of non-rigid articulated objects with multiple independently moving cameras. We demonstrate a hierarchical framework that uses the Hilbert-Schmidt independence criterion, to estimate first the shape and appearance, then the pose, then the activity of the dominant actor. At each level different features are employed, however the same kernel function and the same Hilbert-Schmidt norm are used to estimate the independence. 

The most elementary level involves a single static camera tracking a single moving non-articulated, rigid object. This requires only that the appearance be tracked. Note that the appearance is defined as the combination of the gray-level intensity, the distance function, and the motion. This example is essentially TLD, which is a very robust and accurate tracker. I have made a contribution to show that my algorithm is an extension to TLD in this case. My tracker models a target as a collection of points. Each point is naturally endowed with x, y coordinates. The features for each point are the intensity at that pixel, the distance to the nearest edge, and the motion of that point from the previous frame. In each frame we find the set of points that maximises the HSIC with the reference (as well as maximising the self similarity and minimising the similarity with the negative training example). It is important to note that the HSIC \'fails\' for uniform distributions.

\begin{enumerate}
\item{\textbf{Introduction}}
  \begin{enumerate}
    \item{Mixed reality}
        \item{Pose, shape, appearance}
        \item{Articulated objects}
        \item{Markerless motion capture}
        \item{Appearance modelling}
        \item{Activity recognition}
        \item{Independently moving cameras}
  \end{enumerate}
\item{\textbf{Hilbert Schmidt independence criterion}}
  \begin{enumerate}
    \item{Euclidean spaces}
    \item{Functional analysis}
    \item{Hilbert spaces}
    \item{Reproducing property}
    \item{RKHS}
    \item{Kernels}
    \item{Covariance Measures}
    \item{HSIC measure}
  \end{enumerate}
\item{\textbf{Information theory}}
  \begin{enumerate}
    \item{Entropy / uncertainty}
        \item{Mutual information}
        \item{Pose estimation}
          \end{enumerate}
\item{\textbf{Appearance tracking}}
\item{\textbf{Pose estimation}}
  \begin{enumerate}
    \item{Bottom-up}
        \item{Top-down: Analysis by synthesis}
        \item{Calibration free - motion parameter estimation}
          \end{enumerate}
\item{\textbf{Activity recognition}}
  \begin{enumerate}
    \item{Temporal synchronisation}
      \item{Dynamic time warping}
        \end{enumerate}
\end{enumerate}